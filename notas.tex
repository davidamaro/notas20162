Nuclear y subnuclear.
=====================
En 1897 Bequerel descubre que al dejar la placa de cierto
material no había sol y lo guardó en su escritorio cuando lo
sacó no se acostumbraba a revelar las placas de rayos X que
se habían descubierto dos años antes, él las reveló y
descubrió la radiación.

Después llegó el modelo de Rutherford del que sólo habló
para explicar el modelo.

Chadwick encontró el neutrón. Esto es lo más importante de
toda la historia ya que se mencionó que antes del
descubrimiento del neutrón existía el siguiente problema:
El isótopo de ${}^{14}_7 N_7$ se sabía que tenía una masa
igual a 14 veces la del protón y una carga igual a 7, por lo
que se tenían $14+7=21$ por lo que tenía espín entero pero se
había medido lo contrario. Fue hasta que se tomó en cuenta
que el nucleo d edicha partícula estaba formado por siete
protones y siete neutrones cuando las cuentas dieron.

La vida media del protón es mayor que 10 a la veintiseis
años mientras que la vida media del neutron es de 15
minutos.

Cuando se graficaba energía del electrón contra el número de
electrones se obtenía una distribución, distinto a la
constante que se esperaba. Pauli no pudo ir a una
conferencia pero envió una carta en la que proponía la
existencia de una nueva partícula "fundamental" que podía
explicar esa distribución ya que lo que se estaba graficando
era la diferencia entre la energía del protón y neutrón pero
esta se repartía entre el electrón y el neutrino.

Habló sobre el decaimiento $\beta$ antes de decirnos cuál
es. El decaimiento $\beta$ se trata de que un neutron decae
en un protón un electrón y el antineutrino del electrón.

Para una partícula libre se tiene que se puede perder un
neutrón y ganar un neutron más lo que se dijo. La reacción
"contraria", i.e., que se pierda un protón y se gane un
neutrón sólo se puede dar en el núcleo para que existe
conservación de la energía, esta se obtiene de la energía de
amarre. El primero es el decaimiento $\beta^+$ y el segundo
es $\beta^-$.

Nos explicó la fisión nuclear O - o + o + 3n y esos
neutrones actúan como catalizadores que permiten continuar.
Fermi tuvo el logro del decaimiento $\beta$ y de poder
controlar dicha reacción.

Existen dos modelos desarrollados en los 40s 50s: capaz y
colectivos.

El elemento más pesado y más estable es el
${}^{205}_{52}Pb_{156}$. Existen más o menos 300 núcleos
estables y de 7000 a 8000 inestables.

En una gráfica de $Z$ vs $N$ se observa que los que están
más cerquita de la identidad son núcleos estables.

Para estudiar los nucleos se sigue el siguiente proceso:
$a A -- B b$ en donde $A$ es el blanco $a$ el proyectil, $B$
el resultado y $b$ el detector. Se estudia el detector para
conocer el blanco. La notación es $A(a,b)B$.

Los estados se clasifican por momento angular y paridad.

La masa se determina como $N(A,Z) = 2 M_p c^2 + N M_n c^2 -
BE(A,A)$ con lo último la energía de amarre; en caso de ser
cero no existe el núcleo.

Notas FAMC
==========

Un atomo exótico se le cambia el protón por un muón que es
más pesado, por lo que el neutron está más cerca.

Un problema que no sé si ya se resolvío es que a niveles de
energía se obtiene un radio del protón distinto al que se
obtuvo con métodos de dispersión.

Los relojes más precisos que los que utilizan cesio, usan
átomos ionizados a temperaturas menores de un Kelvin y
pulsos laseres cortos.

Se estudian pulsos de femtosegundos porque esta es la
duración de la formación de una molécula
¿Cómo se forma una molécula?
