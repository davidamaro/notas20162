%==================
%Bitácora travis ci
%==================
\section{Travis CI}

Se le da permiso de acceso a un usuario de github.
Utilizando su interfaz se elige sincronizar un repositorio.
Después se le tiene que elegir una manera de compilar en la
que se elige el lenguaje y configuración.

Compila cosas y ve si compilan bien basado en el exit code
de un comando especificado con la entrada script de su
archivo de configuración.

¿Se puede evitar la elaboración del archivo configure?

%=================================
%Notas física nuclear y subnuclear
%=================================
\section{Física nuclear y subnuclear}

Habló de:

espín, suma de momento angular, eigenvalores de estos
operadores, diagramas de feynman para procesos físicos,
principio de exclusión de pauli, encontrar expresiones para
el rango de una interacción basándose sobre su masa, algunas
partículas elementales, las partículas que median las
fuerzas.

Primero pintó las líneas que mostraban los niveles a partir
de $L^p$, con $L$ el momento angular y $p$ la paridad. Nos
comentó que la paridad es un operador que conmuta con el
hamiltoniano porque cuando $V\neq V(\vec{r})$ entonces, no
importa si tomamos $\vec{r}$ o $-\vec{r}$. Nos hizo un
dibujito para que vieramos que en coordenadas esféricas el
cambio de $r\to -r$ se veía así $r,\vartheta,\varphi \to r,
\pi - \vartheta, \pi + \varphi$. Utilizando esto vimos que
,para el átomo de H, la paridad define una fase que depende
del momento angular, $(-1)^L$.

Nos platicó que Goudsmith y Uhlenbeck observamos que el
patrón de interferencia del electrón se desdoblaba sólo en
dos, lo que quería decir que $2 l + 1 = 2 \Rightarrow l =
1/2$ y se lo asociaron a una propiedad intrínseca de la
materia.

Nos dijo que para que los fermiones tienen asociadas
funciones de onda antisimétricas y los bosones simétricas.
Esto provoca que cumplan el principio de exclusión de Pauli,
ya que esto nos permite ver que los fermiones no pueden
tener el mismo nivel ya que no cumplirían esto.

Nos comentó que la forma de medir momento angular es
utilizar un acoplamiento $\vec{b}\cdot\vec{L}$.

