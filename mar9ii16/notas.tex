\section{Física nuclear y subnuclear}
Lo más importante fue el hecho de que la energía se le
agrega un término complejo, $-i \frac{1}{2} \Gamma$, para
que la probabilidad sea, $\vert \psi(t) \vert^2 = e^{-\Gamma
t}$.

El profesor ante mi pregunta de cual era la interpretación
de dicha energía compleja me repitió que era para que
correspondiera al decaimiento. Luego le pregunté si era el
caso de un operador no hermitiano y contestó que este caso
la ecuación de Schrödinger no se aplica para decaimientos
sólo para partículas estables. También dijo algo sobre la no
hermiticidad del operador.

Me surgieron las siguientes preguntas: ¿Qué ecuación
resuelve los decaimientos? ¿Existe dicha ecuación? ¿Es
posible escribir un hamiltoniano para estas cosas? ¿Cómo se
estudia un decaimiento? ¿Esas energías corresponden a los
eigenvalores de un operador?

Habló de: ley de decaimiento exponencial, gráfica de
energía, la gamma, tiempo promedio (hacer cuenta), vida
media, distribución de Breit-Wigner, transformada de
Fourier.

Primero escribió la distribución de número de eventos contra
la energía. Esta gráfica es tal que en un punto tiene un
máximo, al que le puso $E_0$. En la misma gráfica a
determinada altura de ella puso un \textbf{ancho} que lo
denotó por $\Gamma$. Y dijo que este ancho estaba
relacionado con la \textbf{vida media} de la partícula, la cual
definió como el \emph{el intervalo de tiempo en el que queda
sólo la mitad de materia}. La solución a la ecuación de
decaimiento es $N_0 \exp(-t \lambda)$ por lo que $t_{1/2} =
\frac{\log 2}{\lambda}$. También definió la \textbf{vida
media \textsl{promedio}} como:
\[
  \tau = \frac{\int N(t) t\, dt}{\int N(t) \, dt} =
  \frac{1}{\lambda}.
\]

\subsection{Ecuación de Schrödinger y distribución de
Breit-Wigner}

La ecuación de Schrödinger es $-i\hbar\partial_t \psi = H
\psi$, cuya solución es $\psi_0 \exp(-E t /\hbar)$. La
probabilidad destá dada por $\vert \psi \vert^2 = \vert
\psi_0 \vert^2$ por lo que \textbf{no corresponde con lo
observado para decaimientos}. Se propone intercambiar $E \to
E_0 - i\frac{1}{2} \Gamma$. Esto resulta en que ahora $\vert
\psi \vert^2 = \vert \psi_0\vert^2 \exp(-t \Gamma)$ que
ahora sí queda bajo la situación de que $\Gamma = \lambda
\hbar$.

\subsection{Transformada de Fourier}

La transformada de Fourier está definida así:

\begin{align*}
  f(t) &= \frac{1}{\sqrt{2\pi}} \int d\omega \, g(\omega)
  \exp(-i\omega t),\\
  g(\omega) &= \frac{1}{\sqrt{2\pi}} \int dt \, f(t)
  \exp(i\omega t).
\end{align*}
Le aplicamos esto a $\psi(t) \to f(t)$ para obtener
\[g(\omega) = \psi_0 \frac{1}{\sqrt{2\pi}} \int dt\,\psi_0
\exp\Bigl(i(\omega - E_0/\hbar)\Bigr)t \exp(-\Gamma
t/2\hbar).
\]
Dado que es sencilla la integral podemos ignorar que se
trata de una integral en el plano complejo y hacerla
considerando a $i$ como un numerito más. Por lo que obtener
$g(\omega)$ es inmediato. Para obtener la probabilidad ahora
se usa que $P(E=\hbar\omega) = \vert g(\omega)\vert^2$ por
un factor de normalización:
%% Se hace la integral y el cociente se multiplica y divide
%% por i hbarra. Diferimos en un signo. Luego recordamos la
%% expresión para el inverso de un número complejo y
%% obtenemos parte de la expresión
\[P(E) = \frac{\Gamma}{2\pi}\frac{1}{(E-E_0)^2 +
(\Gamma/2)^2}.\] Este es la fórmula de
\textbf{Breit-Wigner}. De ella podemos ver que el máximo se
alcanza en $E_0$ y que la mitad de la altura se alcanza para
las energías $E_0 \pm \Gamma/2$, de ahí que $\Gamma$ se
considerer la cintura.

Este distribución resultado nos permite calcular \emph{vidas
medias} super pequeñas a partir de la distribución. % el
% artículo de Breit-Wigner está bien loco

\subsection{Ejercicio}
Consideramos el mesón $\rho$ que tiene $\Gamma = 150
\mathrm{MeV}$. La vida media promedio está dada por $\tau =
\hbar/\Gamma = \hbar c / \Gamma c$. El valor de $\hbar c =
197.327 \mathrm{MeV fm}$ y el número de abajo, sabiendo que
$c = 3\times10^{8} \mathrm{ms}^{-1}$ y un fermi $1
\mathrm{fm} = 1\times10^{-15}\mathrm{fm}$. El valor que se
calcula es la masa invariante o masa en reposo $E = mc^2$.

Se escribió una tabla con los valores de vida media promedio
y ancho.
%%%%%%%%%%%%%%%%%%%%%%%%
%fuente      vmp anchura
%%%%%%%%%%%%%%%%%%%%%%%%
%hadrónicos -23  100-200 MeV
%em         -18  10-5 Mev=1.5 KeV
%débil      -10

|fuente|vida media promedio | anchura|
|------|------------------- | -------|
|hadrónicos|-23|100-200 MeV|
|em|-18|10-5 MeV=1.5 KeV|
|débil|-10|-|
